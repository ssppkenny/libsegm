\documentclass{article}
\usepackage{multicol}
\title{Page Segmentation Algorithm}
\author{Sergey Mikhno}
\date{April 2022}

\begin{document}
\maketitle

\begin{multicols}{2}
\section{Problem Description}
Sometimes we have books which are scanned or photographed. To be able to read such books on a mobile device we have to make pages readable. We have to find all the printed symbols and separate them from images. After that the page is redrawn in a larger scale and all the symbols are reflowed.

A scanned page is translated into an array of pixel intesities. For the reflowed page we create a new array. Below we have the steps necessary to reflow the page image.

\begin{enumerate}
\item Open an image file as a grayscale array.
\item Threshold the image with OTSU and BINARY\_INV.
\item Find all components containing connected non-zero pixels.
\item For every component find bounding rectangles.
\item Eliminate all rectangles contained inside others.
\item Join all intersecting rectangles.
\item Make a histogram of rectangle heights.
\item The hight with the highest frequency is text height.
\item Mark or remove all components with the hight or width \textless than $ 5 \times $ most frequent text hight.
\item For every rectangle find a neighboring one to the right, it is a nearest rectangle  intersecting or being inside of the interval of heights [y, y + height], where y is the coordinate of the component's left corner.
\item Create a graph, add an edge between all components and their immediate right neighbors.
\item Find connected components in that graph.
\item Join intersecting ones.
\item Those components are text lines.
\item Sort the text lines and symbols inside them.
\end{enumerate}

\end{multicols}
	

	
\end{document}
	